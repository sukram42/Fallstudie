\section{Konzept \textnormal{\textsf{\small{Niklas Schuster}}}}
\subsection{Spieldynamik}
Die Spieldynamik des zu entwickelnden Planspiels wird gesteuert und bestimmt durch ein Double-Layer Ansatz welcher das Kernelement des Spiels darstellt und folgende logische Schichten beinhaltet:
\begin{enumerate}
\item Die erste Schicht beinhaltet sämtliche funktionale Spielinhalte wie die einzelnen Abteilungen und Features, die für die Spieler spielbar sind.
\item Die zweite Schicht umfasst ein umfangreichen Zahlenpool, welcher aus monetären und nicht monetären Kennzahlen besteht und bestimmt ist die Entscheidungen des Spielers effektiver und realistischer auf das Spielgeschehen abzubilden.
\end{enumerate}
Entscheidend hierbei ist, dass jede Aktion im Spiel die der Spieler ausführt direkten Einfluss auf den Erfolg des zu führenden Unternehmens hat. Dies wird an folgendem Beispiel deutlich:
\par In der Abteilung Human Ressource hat der Spieler die Möglichkeit Mitarbeiter für sein Unternehmen einzustellen. Ein Mitarbeiter ist im Spiel ein Objekt welches verschiedene Attribute aufweist. So bekommt ein Mitarbeiter z.B. ein Gehalt, sowie eine Abteilung die ihm vom Spieler zugewiesen beziehungsweise als Eingabe durch den Spieler frei wählbar ist. Das Gehalt was den Mitarbeitern bezahlt wird beeinflusst der Höhe nach die Kennzahl Mitarbeiterzufriedenheit und andere Kennzahlen aus dem Zahlenpool beeinflusst. Der Durchschnitt aller Kennzahlen ist ein Indikator für die Marge die sich beim Verkauf von Produkten entsteht. Ein Unternehmen mit zufriedenen Mitarbeitern und einem guten Image kann also Produkte zum selben Herstellungspreis teurer Verkaufen. Somit ergeben sich automatisch aus dem Führungsstil des Spielers unterschiedliche Unternehmensstrategien wie die Kostenführerschaft oder Differenzierung. Wenn der Spieler nur einen geringen Durchschnitt aller Kennzahlen erzielt, weil er z.B. seinen Mitarbeitern wenig Geld zahlt und folglich eine geringere Marge erhält, bleibt ihm nur die Möglichkeit seine Produkte billig zu produzieren und durch hohe Absatzmengen erfolgreich zu sein (Kostenführerschaft). Andersherum wäre es effektiver teuer zu produzieren (Differenzierung). Die Abteilung die den Mitarbeitern zugewiesen wird beeinflusst ebenfalls drastisch das Spielgeschehen. Durch viele im Sales beschäftigte Mitarbeiter steigt beispielsweise die Chance einen Deal zu gewinnen und die maximale Anzahl an Kunden die gleichzeitig betreut werden können usw. So lässt sich allein durch das Einstellen von Mitarbeitern sein Unternehmen bewusst in eine bestimmte Richtung steuern.
\par Mit diesem Konzept wird ein Ansatz verfolgt bei dem der Spieler im Verhältnis ein begrenztes Maß an Aktionen ausführen kann, wodurch das Planspiel übersichtlich und intuitiv für den Spieler bleib, aber gleichzeitig durch das Kennzahlen Modell ein hohes Maß an Komplexität und Entscheidungsfreiheit geschaffen wird. Somit kann im Early-Game der Start für den Spieler erleichtert werden und im Mid/End-Game weiterhin erfolgsrelevante Entscheidungen getroffen werden.

\subsection{Echtzeit \& Ranking}
Um die Entscheidungen im Spiel noch erfolgsrelevanter gestalten zu können wird im Spiel auf ein "Echtzeit" System gesetzt. Die Zeit In-Game basiert auf einem Datumssystem wobei ein Tag 16 Minuten in Real-Time entspricht. Die gesamte Spielzeit des Spiels ist beschränkt auf 10 Geschäftsjahre, dabei ist es irrelevant zu welcher Zeit der Spieler mit dem Spiel beginnt. Sobald sich ein Spieler auf dem Server registriert beginnt seine Spielzeit. Nach den 10 Geschäftsjahren ist das Spiel für den jeweiligen Spieler beendet und sein Unternehmen wird in eine Highscoreliste eingetragen. Somit lässt sich das Spiel unabhängig von anderen Spielern frei spielen, wobei sich der Markt und die Konkurrenzsituation aus allen momentan aktiven Spielern ergibt. Durch das Login System ist die Spieleranzahl beliebig skalierbar. Der Faktor der Zeit spielt im Spiel eine Entscheidende Rolle und hebt den planerischen Aspekt des Spiels deutlicher hervor, hierzu ein Beispiel:
\par In der Produktion können Maschinen für die Herstellung von Produkten gekauft werden. Diese haben eine maximale Ausbringungsmenge, welche pro Monat angegeben wird: in diesem Beispiel 30.000 Einheiten pro Monat. Die tatsächlich produzierte Menge werden aber pro Tag ausgeschüttet, was einer Menge von 1.000 Einheiten entspricht die dem Warenlager hinzugefügt werden (Ausgegangen von 30 Tagen im Monat). Das Lieferdatum gegenüber aktiven Kunden ist in den Verträgen immer zum Ende des Monats angesetzt. Momentan ist eine Ausschreibung offen zum 15. des Monats, wobei der Kunde jeweils 50.000 Einheiten pro Monat beziehen will. Der Bestand im Warenlager beläuft sich am 15. des Monats ebenfalls auf 30.000 Einheiten. An dieser Stelle muss der Spieler neben der entstehenden Kosten auch den Faktor der Zeit deutlich in seine Überlegung mit einbeziehen:
\begin{itemize}
   \item Wie viel produziert meine Maschine in den verbleibenden Tagen bis Monatsende?
    \item Ist dann durch den Lagerbestand die Bezugsmenge des Kunden gedeckt?
    \item Kann die Bezugsmenge des Kunden in der verbleibenden Zeit durch eine zusätzliche Maschine gedeckt werden?
    \item Können die dadurch entstandenen Mehrkosten getragen werden?
\end{itemize}
\par Außerdem kann durch den Ansatz einer Echtzeit Simulation zusätzlich das unternehmerische Risiko erhöht werden, wie z.B. Konventionalstrafen bei Nichterfüllung der auszuliefernden Mengen an den Kunden. Spieler die aktiver am Markt sind können sich so ebenfalls einen Zeitvorteil verschaffen, da Spieler die sich eher auf eine Ausschreibung bewerben auch bessere Chancen haben einen Deal für sich zu entscheiden.
