\chapter{Einleitung \textnormal{\textsf{\small{Niklas Schuster}}}}
\section{Aufgabenstellung}
Im Rahmen dieses Projektes, welches im Zuge der Lehrveranstaltung Fallstudie des Moduls Umsetzung der Methoden der Wirtschaftsinformatik erfolgt, soll ein computergestütztes Unternehmensplanspiel entwickelt werden. Dabei soll es sich um ein zu führendes Industrieunternehmen handeln und branchenspezifische Unternehmensprozesse modellhaft simulieren. Weiterhin vorgesehen ist, dass teilnehmende Spieler jeweils ein einzelnes Unternehmen führen sollen und untereinander auf einem Oligopolmarkt konkurrieren. Dafür ist jedes Unternehmen der selben Branche zugeordnet und produziert Produkte in direkter Konkurrenz.
\section{Zielsetzung}
Das Ziel dieser Ausarbeitung ist es, ein funktionierendes Planspiel zu entwickeln, welches in der Programmiersprache Java implementiert wird. Das Planspiel wird den Namen EARTHBOUND tragen und die Führung eines Unternehmens in der Outdoor Branche widerspiegeln, welches auf Rucksäcke, Taschen und ähnliches spezialisiert ist. Als GUI dient dem Spieler ein Web Interface. Dieses wird mit Angular.js und Bootstrap geschrieben. Ziel ist es dem Spieler Unternehmensprozesse in den Bereichen: Human Resources, Produktion, Sales, Marketing und Research zu vermittelt. Besonders großen Augenmerk liegt auf dem planerischen Aspekt des zu führenden Unternehmens sowie das Zeit- und Ressourcen Managements.
