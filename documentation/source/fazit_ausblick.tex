\chapter{Fazit \& Ausblick  \textnormal{\textsf{\small{Christian Sasse}}}}

Am Ende dieses Projektes wurde deutlich, dass man sich, bei der Erstellung des Unternehmensplanspiels zunächst über den Aufbau des Unternehmens klar werden musste. Das heißt, es galt zunächst herauszufinden, welche Abteilungen und Kennzahlen überhaupt benötigt werden würden und welche und wie viele Produktarten produziert und verkauft werden sollten.
Die nächste Schwierigkeit bestand darin, den generellen Spielablauf festzulegen. Hierbei galt es unter anderem die Spielzeit zu definieren.
Die gewonnen Erkenntnisse dienten als erster Entwurf, der im Laufe der Entwicklung immer wieder verändert und angepasst wurde. Hierbei gab es auch die größten Probleme. Einzelne Abteilungen und Objekte, aber auch Kennzahlen mussten immer wieder verändert oder gar neu erstellt werden. Somit war es auch notwendig immer wieder neue Beziehungen zwischen den einzelnen Klassen festzulegen. Diese Probleme zogen sich, wenn auch mit der zeit immer geringer werdender, durch den ganzen Projektzeitraum. Die zu Beginn in der Einleitung definierte Zielsetzung wurde am Ende des Projektes jedoch umgesetzt.

In einer zukünftigen Entwicklung könnte das Unternehmensplanspiel noch weiter ausgebaut werden. Im Unternehmen selbst könnten zum Beispiel noch mehr Produkte zur Produktion angeboten werden.
Um das Planspiel noch realistischer zu gestalten, könnte eine Rohstoffverwaltung implementiert werden. Damit ist gemeint, dass die Spieler, die zur Produktion benötigten Rohstoffe selbst einkaufen müssen. Hierbei könnte dann auch zwischen verschiedenen Qualitätsstufen gewählt werden. Da dies eine erhebliche Erhöhung des Schwierigkeitsgrades mit sich führen würde, wäre es vorstellbar verschiedene, zu Beginn auswählbare Schwierigkeitsstufen einzuführen. Dies wäre aber nur bei einer ausreichend großen Spieleranzahl sinnvoll.

Weiterhin wäre es vorstellbar, den Kontakt zwischen den einzelnen Spielern zu ermöglichen. Hierfür könnte man zum Beispiel einen Chat implementieren. Im Zuge dessen wäre es auch überlegenswert, einen \enquote{Handel} zwischen den Spielern einzuführen. So könnte man sich zum Beispiel Resourcen zum Kauf anbieten oder sogar fertige Produkte.

Man sieht, dass eine zukünftige Entwicklung viele Richtungen einschlagen könnte. Vor allem im Bereich der Interaktion zwischen den Spielern wäre noch sehr viel möglich.

